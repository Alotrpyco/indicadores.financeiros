	
\documentclass[1pt,a4paper]{article}

% Pacotes
\usepackage{amsmath}
\usepackage{geometry}
\usepackage[utf8]{inputenc}
\usepackage[T1]{fontenc}
\usepackage[brazil]{babel}
\usepackage{booktabs}
\usepackage{longtable}
\usepackage{array}
\usepackage[table]{xcolor} % Para zebrado
\usepackage{lipsum} % Apenas para texto de exemplo
\usepackage{setspace} % Para espaçamento

% Espaçamento entre linhas
\onehalfspacing

%configuração de layout
\geometry{
	top=2.5cm,
	bottom=2.5cm,
	left=3cm,
	right=2cm
}
\setlength{\parindent}{1.25cm}


% Documento
\begin{document}
	
	% Título
	\begin{center}
		\LARGE \textbf{Indicadores Financeiros: Análise do Balanço Patrimonial da S/A Usina Coruripe Açúcar e Álcool (2023–2024)  
		 } \\[1cm]
		
		% Autores
		\large
		Sérgio Ricardo Vieira Torres Silva\\
		\texttt{sergio.torres@feac.ufal.br}\\
		UFAL\\[0.5cm]
		
		Talysson Rafael Lima Ferreira\\
		\texttt{talysson.ferreira@feac.ufal.br}\\
		UFAL\\[0.5cm]
		
		Alex dos Santos Júnior\\
		\texttt{alex.junior@feac.ufal.br}\\
		UFAL\\[0.5cm]
		
		
	\end{center}
	
	\vspace{1cm}
	
	% Resumo
	\noindent \textbf{Resumo:} O presente trabalho foi desenvolvido com o objetivo de analisar, por meio de indicadores, o desempenho através do balanço patrimonial. Portanto, concentramos esforços em desenvolver um estudo dos indicadores financeiros: liquidez corrente (LC), margem bruta (MB), retorno sobre investimento (ROI), quociente de imobilização do capital próprio (QICP), taxa de rentabilidade do capital próprio (TxRCP), margem operacional (MO), grau de endividamento geral (GEG), composição de endividamento (CE), earnings before interest and taxes (EBIT), ou, em português, conhecido como Lucro Antes dos Juros e Impostos (LAJIR).
	
	\section{INTRODUÇÃO}
	\setlength{\parindent}{1.5cm}
	
	\hspace*{1.5cm} O Grupo S/A Usina Coruripe, uma das maiores e mais tradicionais empresas do setor sucroenergético no Brasil, representa um caso de estudo relevante para a análise de indicadores de desempenho financeiro. Com uma história sólida e uma presença significativa no mercado, a empresa se destaca não apenas pela sua capacidade produtiva, mas também por sua relevância econômica e social para as regiões onde atua, em Alagoas, Minas Gerais e São Paulo. A avaliação de sua saúde financeira e capacidade de gestão é essencial para compreender sua sustentabilidade e seu potencial de crescimento. Nesse contexto, a análise detalhada dos seus balanços financeiros e demonstrativos de resultados se torna uma ferramenta fundamental para mensurar seu desempenho e identificar pontos fortes e fracos, fornecendo uma visão clara de sua situação atual.
	
	As demonstrações financeiras são os insumos básicos no processo de análise dos balanços e podem ser definidas como relatórios contábeis elaborados periodicamente pelas empresas para demonstrar sua situação econômico-financeira (Assaf Neto, 2012). Sendo assim, para aprofundar a análise da Usina Coruripe, um conjunto de indicadores financeiros foi selecionado, oferecendo uma visão de sua gestão e performance. A seleção abrange métricas de liquidez, como a Liquidez Corrente, que avalia a capacidade da empresa de honrar suas obrigações de curto prazo, e indicadores de rentabilidade, como o ROI (Retorno sobre o Investimento) e a Taxa de Rentabilidade do Capital Próprio, essenciais para mensurar a eficiência na geração de lucros. Adicionalmente, a Margem Bruta e a Margem Operacional permitem entender a eficiência na gestão de custos e despesas. Para avaliar a estrutura de capital e o nível de alavancagem, foram escolhidos o Grau de Endividamento Geral e a Composição de Endividamento (curto prazo), além do Quociente de Imobilização de Capital Próprio, que indica o quanto do capital próprio está comprometido com ativos de longo prazo. Por fim, o EBIT (Lucro antes de Juros e Impostos) foi incluído para refletir a rentabilidade operacional antes de influências financeiras e fiscais. Esses indicadores, extraídos do Balanço Patrimonial e da DRE da empresa, serão a base para a análise. A Usina Coruripe Açúcar e Álcool foi escolhida para que possamos analisar seu desempenho através do balanço patrimonial, com base nos anos de 2023 e 2024. Além disso, foram escolhidos dez indicadores para avaliarmos seu desempenho.
	
	O balanço patrimonial da Usina Coruripe apresenta a posição financeira anualmente. Portanto, ele equilibra os ativos, ou seja, o que a usina possui, em relação ao seu financiamento, que é a dívida e o patrimônio líquido. Na Tabela 1, é possível ver detalhadamente os ativos circulantes e não circulantes. Por outro lado, na Tabela 2, é possível ver detalhadamente o passivo circulante, não circulante e o patrimônio líquido. Além disso, na Tabela 3, foi necessário trazer a demonstração de resultados, pois indicadores como EBIT (Lucro antes de Juros e Impostos) e MO (Margem Operacional) foram alguns dos indicadores escolhidos por nossa equipe.
	
	\section{INDICADORES FINANCEIROS} 
	\setlength{\parindent}{1.5cm} 
	\hspace*{1.5cm} Os indicadores são ferramentas fundamentais para avaliar a saúde financeira e a performance operacional de uma empresa. O Quociente de Imobilização do Capital Próprio (QICP) verifica como os recursos dos sócios são investidos em bens duradouros. Outro indicador, a Taxa de Rentabilidade do Capital Próprio (ROE), aponta o quão bem a empresa transforma seus lucros em ganhos para os investidores. Carvalho (2008) observa esse aspecto ao estudar cooperativas. A Margem Operacional (MO), por sua vez, explicita a aptidão em transformar vendas em lucro, evidenciando o controle dos gastos e custos do dia a dia. Neves (2017) mostrou em seus estudos que analisar esses indicadores é vital, pois permite entender a fundo a situação da empresa e avaliar, de forma clara, como ela está se saindo. Essa compreensão ampla é a base para decidir os próximos passos. Utilizar métricas como o ROE, por exemplo, é importantíssimo para saber o retorno sobre o investimento, um ponto-chave para os acionistas. Em resumo, estudar o QICP, a Tx RCP e a MO dá uma visão completa da força, do lucro e da eficiência da Usina Coruripe Açúcar e Álcool.
	
	Além disso, O Grau de Endividamento Geral (GEG) é um indicador que mede a participação do capital de terceiros no financiamento dos ativos da empresa, ou seja, revela quanto do ativo total é sustentado por dívidas em vez de capital próprio. Esse índice é fundamental para avaliar o nível de dependência financeira da organização em relação a recursos externos, como empréstimos, financiamentos e obrigações com fornecedores. Um GEG elevado significa que uma parcela significativa dos ativos da empresa é financiada por terceiros; por outro lado, um GEG mais baixo indica que a empresa depende menos de dívidas e utiliza mais capital próprio, transmitindo maior solidez e segurança para investidores e credores. 
	
	Ademais, A Composição do Endividamento (CE) avalia a proporção das dívidas de curto prazo em relação ao total das obrigações da empresa. Esse indicador é relevante porque permite identificar se a maior parte do endividamento recai sobre compromissos de liquidez imediata ou se está distribuída ao longo do tempo, em dívidas de longo prazo. O EBIT (Earnings Before Interest and Taxes), é um indicador financeiro que mede a capacidade de geração de resultado de uma empresa apenas pelas suas atividades operacionais, sem considerar o impacto das despesas financeiras (juros de dívidas) e dos tributos sobre o lucro. Ele mostra, portanto, se o negócio é eficiente e lucrativo em sua operação principal, independentemente da forma de financiamento adotada ou da carga tributária incidente. 
	
	\section{INDICADORES, APLICABILIDADES E RESULTADOS}
	\subsection{Liquidez Corrente - LC}
	
	\subsection{Margem Bruta - MB}		
	
	\subsection{Retornos Sobre o Investimanto - ROI}
	
	\subsection{Quociente de Imobilização do Capital Próprio - QICP}
	
	\subsection{Taxa de Rentabilidade do Capital Próprio - TxRCP}
	\hspace*{1.5cm} A TxRCP também conhecida como ROE, da empresa teve uma queda significativa, de 19\% em 2023 para 9\% em 2024.  Essa diminuição é um sinal negativo, indicando que a empresa se tornou menos eficiente em gerar lucro para seus acionistas em 2024. A rentabilidade sobre o investimento dos proprietários do negócio foi reduzida pela metade. Segundo a pesquisa de Juliana Medeiros das Neves (2017), a análise de indicadores como o ROE é crucial, pois demonstra a capacidade da empresa de gerar lucro em relação ao capital investido pelos sócios, sendo um dos fatores mais importantes para os acionistas.
		\begin{center}
			\begin{tabular}{|c|}
				\hline
				\left( \frac{LUCRO}{PAT. LÍQUIDO} \right) \times 100 = Tx \, RCP \\
				\hline
			\end{tabular}
		\end{center}
		
	\subsection{Margem Operacional - MO}
		\begin{center}
		\begin{tabular}{|c|}
			\hline
			\left( \frac{LUCRO OPERACIONAL}{RECEITA LÍQUIDA} \right) = MO \\
			\hline
		\end{tabular}
	\end{center}
	
	
	\subsection{Grau de Endividamento - CEG}
	
	\subsection{Composição de Endividamento - CE}
	
	\subsection{Earnings Before Interest and Taxes - EBIT}
	
	\section{CONSIDERAÇÕES FINAIS}
	
	\newpage
	% Definir cor clara e cor escura para destaques
	\definecolor{lightgray}{gray}{0.95}
	\definecolor{darkgray}{gray}{0.85}
	
	\centering \textbf{ANEXOS}
		
		\begin{center}
			\textbf{\Large S/A Usina Coruripe Açúcar e Álcool}\\
			\textbf{Balanço Patrimonial referente ao período de 31 de março de 2023 a 31 de dezembro de 2023 e de março de 2024 a 31 de dezembro de 2024}\\
			(Valores expressos em milhares de reais)
		\end{center}
		
		
		
		%==================== ATIVO ====================%
		\centering\section*{Ativo}
		Tabela 1 - Balanço patrimonial referente ao ano 2023 e 2024
		\rowcolors{2}{lightgray}{white} % Linhas zebras
		\begin{longtable}{p{6cm}r r r r }
			\toprule
			& \multicolumn{2}{c}{\textbf{Consolidado}} & \multicolumn{2}{c}{\textbf{Consolidado}} \\
			\cmidrule(lr){2-3} \cmidrule(lr){4-5}
			& 31/03/2023 & 31/12/2023 & 31/03/2024 & 31/12/2024 \\
			\midrule
			\endhead
			%---- Circulante
			\textbf{Circulante} & & & & \\
			Caixa e equivalentes de caixa & 246.682 & 390.862 & 769.658 & 1.155.469 \\
			Aplicações financeiras & 127.021 & 99.145 & 155.924 & 158.542 \\
			Contas a receber de clientes & 174.933 & 102.282 & 144.124 & 106.947 \\
			Estoques & 844.828 & 162.191 & 843.598 & 213.391 \\
			Adiantamentos a fornecedores & 248.367 & 217.172 & 250.313 & 210.817 \\
			Ativos biológicos & 525.165 & 486.996 & 0,00 & 562.940 \\
			Tributos a recuperar & 140.893 & 171.546 & 154.057 & 146.499 \\
			IR e CS a recuperar & 21.897 & 38.494 & 0,00 & 90.661 \\
			Partes relacionadas & 16.005 & 28.824 & 31.656 & 20.526 \\
			Instrumentos financeiros derivativos & 50.883 & 13.643 & 275.829 & 0,00 \\
			Outros direitos & 98.340 & 85.385 & 88.042 & 50.523 \\
			\rowcolor{darkgray}\textbf{Total do ativo circulante} & \textbf{2.495.014} & \textbf{1.796.540} & \textbf{2.395.349} & \textbf{2.733.072} \\
			\midrule
			%---- Não Circulante
			\textbf{Não circulante} & & & & \\
			Aplicações financeiras & 43.203 & 68.035& 8.478 & 1.525 \\
			Adiantamentos a fornecedores & 195.663 & 194.071 & 142.049 & 149.632 \\
			Partes relacionadas & 105 & 0,00 & 18.432 & 4.431 \\
			Tributos a recuperar & 4.098 & 5.052 & 5.142 & 4.431 \\
			IR e CS diferidos & 508.127 & 41.218 & 508.127 & 41.218 \\
			Instrumentos financeiros derivativos & 0,00 & 21.535 & 75.821 & 13.392 \\
			Depósitos judiciais & 4.524 & 6.251 & 6.974 & 6.391 \\
			Outros créditos/outros direitos & 4.035.181 & 4.216.551 & 4.467.192 & 4.273.838 \\
			Investimentos & 28.224 & 31.751 & 56.523 & 33.193 \\
			Imobilizado & 2.034.027 & 2.193.618 & 2.457.347 & 2.369.749 \\
			Intangível & 3.853 & 6.023 & 6.684 & 0,00 \\
			Direito de uso & 1.723.721 & 1.162.397 & 1.204.369 & 1.246.056 \\
			\rowcolor{darkgray}\textbf{Total do ativo não circulante} & \textbf{8.072.494} & \textbf{7.905.389} & \textbf{9.018.343} & \textbf{8.845.609} \\
			\midrule
			\rowcolor{darkgray}\textbf{Total do Ativo} & \textbf{9.869.034 } & \textbf{10.400.403} & \textbf{12.313.692} & \textbf{10.878.562} \\
			\bottomrule
		\end{longtable}
		
		
		
		
		\newpage
		%==================== PASSIVO E PL ====================%
		\centering\section*{Passivo e Patrimônio Líquido}
		Tabela 2 - Balanço patrimonial referente ao ano 2023 e 2024
		\rowcolors{2}{lightgray}{white} % Linhas zebras
		\begin{longtable}{p{6cm} r r r r}
			\toprule
			& \multicolumn{2}{c}{\textbf{Consolidado}} & \multicolumn{2}{c}{\textbf{Consolidado}} \\
			\cmidrule(lr){2-3} \cmidrule(lr){4-5}
			& 31/03/2023 & 31/12/2023 & 31/03/2024 & 31/12/2024 \\
			\midrule
			\endhead
			%---- Passivo Circulante
			\textbf{Passivo Circulante} & & & & \\
			Fornecedores & 200.066 & 539.767 & 621.204 & 335.828 \\
			Empréstimos e financiamentos & 904.387 & 1.595.291 & 1.743.158 & 1.295.309 \\
			Arrendamento a pagar & 146.348 & 88.011 & 166.196 & 145.323 \\
			Parceria agrícola a pagar & 182.891 & 190.933 & 196.693 & 226.012 \\
			Salários e encargos sociais & 76.272 & 97.077 & 105.753 & 81.723 \\
			Tributos a recolher & 25.137 & 34.636 & 32.455 & 34.256 \\
			IR e CS a pagar & 89,00 & 732,00 & 803.311 & 450.467 \\
			Adiantamento de clientes & 216.574 & 607.742 & 88.289 & 139.702 \\
			Compromissos com contrato de energia & 77.669 & 161.993 & 227.293 & 98.497 \\
			Instrumentos financeiros derivativos & 1.724 & 95.803 & 0 & 9.491 \\
			Outras obrigações & 16.637 & 6.800 & 22.236 & 9.491 \\
			\rowcolor{darkgray}\textbf{Total do passivo circulante} & \textbf{1.847.794} & \textbf{3.418.785} & \textbf{4.036.568} & \textbf{2.787.400} \\
			\midrule
			%---- Passivo Não Circulante
			\textbf{Passivo Não Circulante} & & & & \\
			Empréstimos e financiamentos & 2.737.544 & 2.257.269 & 2.784.617 & 3.050.316 \\
			Arrendamento a pagar & 843.717 & 347.909 & 418.251 & 456.455 \\
			Parceria agrícola a pagar & 556.067 & 508.406 & 599.871 & 548.577 \\
			Tributos a recolher & 168.868 & 180.305 & 176.765 & 12.485 \\
			Instrumentos financeiros derivativos & 44.327 & 0,00 & 13.392 & 75.821 \\
			Adiantamento de clientes & 254.296 & 108.503 & 532.633 & 268.787 \\
			Compromissos com contrato de energia & 140.355 & 33.233 & 25.419 & 19.945 \\
			IR e CS diferidos & 35.745 & 118.395 & 0,00 & 99.316 \\
			Provisões para contingências & 73.120 & 59.156 & 8.672 & 10.166 \\
			Outras obrigações & 487.711 & 509.350 & 516.787 & 564.359 \\
			\rowcolor{darkgray}\textbf{Total do passivo não circulante} & \textbf{5.341.750} & \textbf{4.122.526} & \textbf{5.175.723} & \textbf{5.106.227} \\
			\midrule
			\rowcolor{darkgray}\textbf{Total do Passivo} & \textbf{7.189.544} & \textbf{7.541.311} & \textbf{9.212.291} & \textbf{7.893.627} \\
			\midrule
			
			%---- Patrimônio Líquido
			\textbf{Patrimônio Líquido} & & & & \\
			Capital social & 408.845 & 408.845 & 867.567 & 867.567 \\
			Ações em tesouraria & (1.215) & (1.215) & (1.215) & (1.215) \\
			Ajuste de avaliação patrimonial & 26.987 & 104.773 & 37.464 & 248.494\\
			Reservas de lucros & 2.244.873 & 2.249.326 & 2.011.623 & 1.996.759 \\
			Lucros acumulados & 0,00 & 97.363 & 0.00 & 655.596 \\
			\rowcolor{darkgray}\textbf{Total do patrimônio líquido} & \textbf{2.679.490} & \textbf{2.859.092} & \textbf{2.915.439} & \textbf{3.270.213} \\
			\midrule
			\rowcolor{darkgray}\textbf{Total do Passivo e PL} & \textbf{9.869.034} & \textbf{10.400.403} & \textbf{10.878.562} & \textbf{12.313.692} \\
			\bottomrule
		\end{longtable}
		
\newpage
		
		\begin{center}
			\textbf{\Large S/A Usina Coruripe Açúcar e Álcool}\\
			\textbf{Demonstração do resultado períodos de três e nove meses findos, referente a 31 de dezembro de 2023 a 31 de dezembro de 2024.}\\
			(Valores expressos em milhares de reais, exceto quando indicado de outra forma)
		\end{center}
	
		\centering\section*{Demostração do Resultado}
		Tabela 3 - Demostração do Resultado referente ao ano 2023 e 2024
		\rowcolors{2}{lightgray}{white} % Linhas zebras
		\begin{longtable}{p{6cm}r r r r }
			\toprule
			& \multicolumn{2}{c}{\textbf{31 de dezembro de 2023}} & \multicolumn{2}{c}{\textbf{31 de dezembro de 2024}} \\
			\cmidrule(lr){2-3} \cmidrule(lr){4-5}
			& Trimestre & 9 meses & Trimestre & 9 meses \\
			\midrule
			\endhead
			Receitas operacional líquida & 1.301.446 & 3.075.471 & 1.435.005 & 3.493.657 \\
			Custos dos produtos vendidos & (853.218) &(2.006.538) & (1.071.095)& (2.473.281) \\
			\cmidrule(lr){2-3} \cmidrule(lr){4-5}
			\rowcolor{darkgray} \textbf{Lucro Bruto} & 448.228 & 1.068.993 & 363.910 & 1.020.376 \\
			\cmidrule(lr){2-3} \cmidrule(lr){4-5}
			Despesas com vendas & (50.981) & (160.337) & (67.124) &(209.488) \\
			Despesas gerais e administrativas & (58.822) & (177.119) & (56.407) & (180.920) \\
			Resultado de participação societária & 1.294 & 3.526 & 1.462 & 154.671 \\
			Outras receitas (despesas) operacionais, líquidas & (1.480) & (21.351) & 1.462 & 154. 671 \\
			\cmidrule(lr){2-3} \cmidrule(lr){4-5}
			\rowcolor{darkgray}\textbf{Lucro operacional} & 338.239 & 713.652 & 243.540 & 788.999 \\ 
			\cmidrule(lr){2-3} \cmidrule(lr){4-5}
			Receitas financeiras & 132.190 & 472.916 & 519.216 & 1.076.913 \\
			Despesas financeiras & (294.469) & (1.004.791) & (769.586) & (1.676.735)\\
			\cmidrule(lr){2-3} \cmidrule(lr){4-5}
			\rowcolor{darkgray}\textbf{Resultado financeiro} & (162.279) & (531.875) & (250.370) & (599.822) \\
			\cmidrule(lr){2-3} \cmidrule(lr){4-5}\\
			\rowcolor{darkgray}\textbf{Lucro (prejuízo) antes do imposto de renda e da contribuição social}  & 175.960 & 181.777 & (6.830) & 189.177 \\
			\cmidrule(lr){2-3} \cmidrule(lr){4-5}\\
			Imposto de renda e contribuição social correntes & (570) & (2.313) & (688) & (1.976)\\
			Imposto de renda e contribuição social diferidos & (27.673) & (39.045) & (22.181) & (462.942) \\
			\cmidrule(lr){2-3} \cmidrule(lr){4-5}
			\rowcolor{darkgray}\textbf{Resultado do período} & \textbf{147.717} & \textbf{140.419} & \textbf{29.699} & \textbf{650.143}\\
			\cmidrule(lr){2-3} \cmidrule(lr){4-5}\\
			Lucro (prejuízo) básico e diluído por ação & 105.51 & 100.30 & (21.21) & 464.39 \\
			\cmidrule(lr){2-3} \cmidrule(lr){4-5}
			
				
		\end{longtable}

\end{document}
