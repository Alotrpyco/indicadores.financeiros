	
\documentclass[1pt,a4paper]{article}

% Pacotes
\usepackage{amsmath}
\usepackage{geometry}
\usepackage[utf8]{inputenc}
\usepackage[T1]{fontenc}
\usepackage[brazil]{babel}
\usepackage{booktabs}
\usepackage{longtable}
\usepackage{array}
\usepackage[table]{xcolor} % Para zebrado
\usepackage{lipsum} % Apenas para texto de exemplo
\usepackage{setspace} % Para espaçamento

% Espaçamento entre linhas
\onehalfspacing

%configuração de layout
\geometry{
	top=2.5cm,
	bottom=2.5cm,
	left=3cm,
	right=2cm
}
\setlength{\parindent}{1.25cm}


% Documento
\begin{document}
	
	% Título
	\begin{center}
		\LARGE \textbf{Indicadores Financeiros: Análise do Balanço Patrimonial da S/A Usina Coruripe Açúcar e Álcool (2023–2024)  
		 } \\[1cm]
		
		% Autores
		\large
		Sérgio Ricardo Vieira Torres Silva\\
		\texttt{sergio.torres@feac.ufal.br}\\
		UFAL\\[0.5cm]
		
		Talysson Rafael Lima Ferreira\\
		\texttt{talysson.ferreira@feac.ufal.br}\\
		UFAL\\[0.5cm]
		
		Alex dos Santos Júnior\\
		\texttt{alex.junior@feac.ufal.br}\\
		UFAL\\[0.5cm]
		
		
	\end{center}
	
	\vspace{1cm}
	
	% Resumo
	\noindent \textbf{Resumo:} O presente trabalho foi desenvolvido com o objetivo de analisar, por meio de indicadores, o desempenho através do balanço patrimonial. Portanto, concentramos esforços em desenvolver um estudo dos indicadores financeiros: liquidez corrente (LC), margem bruta (MB), retorno sobre investimento (ROE), quociente de imobilização do capital próprio (QICP), taxa de rentabilidade do capital próprio (TxRCP), margem operacional (MO), grau de endividamento geral (GEG), composição de endividamento (CE), earnings before interest and taxes (EBIT), ou, em português, conhecido como Lucro Antes dos Juros e Impostos (LAJIR).
	
	\section{INTRODUÇÃO}
	\setlength{\parindent}{1.5cm}
	
	\hspace*{1.5cm} O Grupo S/A Usina Coruripe, uma das maiores e mais tradicionais empresas do setor sucroenergético no Brasil, representa um caso de estudo relevante para a análise de indicadores de desempenho financeiro. Com uma história sólida e uma presença significativa no mercado, a empresa se destaca não apenas pela sua capacidade produtiva, mas também por sua relevância econômica e social para as regiões onde atua, em Alagoas, Minas Gerais e São Paulo. A avaliação de sua saúde financeira e capacidade de gestão é essencial para compreender sua sustentabilidade e seu potencial de crescimento. Nesse contexto, a análise detalhada dos seus balanços financeiros e demonstrativos de resultados se torna uma ferramenta fundamental para mensurar seu desempenho e identificar pontos fortes e fracos, fornecendo uma visão clara de sua situação atual.
	
	As demonstrações financeiras são os insumos básicos no processo de análise dos balanços e podem ser definidas como relatórios contábeis elaborados periodicamente pelas empresas para demonstrar sua situação econômico-financeira (Assaf Neto, 2012). Sendo assim, para aprofundar a análise da Usina Coruripe, um conjunto de indicadores financeiros foi selecionado, oferecendo uma visão de sua gestão e performance. A seleção abrange métricas de liquidez, como a Liquidez Corrente, que avalia a capacidade da empresa de honrar suas obrigações de curto prazo, e indicadores de rentabilidade, como o ROE (Retorno sobre o Investimento) e a Taxa de Rentabilidade do Capital Próprio, essenciais para mensurar a eficiência na geração de lucros. Adicionalmente, a Margem Bruta e a Margem Operacional permitem entender a eficiência na gestão de custos e despesas. Para avaliar a estrutura de capital e o nível de alavancagem, foram escolhidos o Grau de Endividamento Geral e a Composição de Endividamento (curto prazo), além do Quociente de Imobilização de Capital Próprio, que indica o quanto do capital próprio está comprometido com ativos de longo prazo. Por fim, o EBIT (Lucro antes de Juros e Impostos) foi incluído para refletir a rentabilidade operacional antes de influências financeiras e fiscais. Esses indicadores, extraídos do Balanço Patrimonial e da DRE da empresa, serão a base para a análise. A Usina Coruripe Açúcar e Álcool foi escolhida para que possamos analisar seu desempenho através do balanço patrimonial, com base nos anos de 2023 e 2024. Além disso, foram escolhidos dez indicadores para avaliarmos seu desempenho.
	
	O balanço patrimonial da Usina Coruripe apresenta a posição financeira anualmente. Portanto, ele equilibra os ativos, ou seja, o que a usina possui, em relação ao seu financiamento, que é a dívida e o patrimônio líquido. Na Tabela 1, é possível ver detalhadamente os ativos circulantes e não circulantes. Por outro lado, na Tabela 2, é possível ver detalhadamente o passivo circulante, não circulante e o patrimônio líquido. Além disso, na Tabela 3, foi necessário trazer a demonstração de resultados, pois indicadores como EBIT (Lucro antes de Juros e Impostos) e MO (Margem Operacional) foram alguns dos indicadores escolhidos por nossa equipe.
	
	\section{INDICADORES FINANCEIROS} 
	\setlength{\parindent}{1.5cm} 
	\hspace*{1.5cm} Os indicadores são ferramentas fundamentais para avaliar a saúde financeira e a performance operacional de uma empresa. Em relação a liquidez corrente (LC), Marquez (2025) descreve esse indicador como a capacidade da empresa de pagar dívidas de curto prazo com recursos do ativo circulante. A Margem Bruta (MB) reflete a eficiência da usina em gerar lucro a partir de suas vendas.  O Quociente de Imobilização do Capital Próprio (QICP) verifica como os recursos dos sócios são investidos em bens duradouros. Outro indicador, a Taxa de Rentabilidade do Capital Próprio (ROE), aponta o quão bem a empresa transforma seus lucros em ganhos para os investidores. Carvalho (2008) observa esse aspecto ao estudar cooperativas. A Margem Operacional (MO), por sua vez, explicita a aptidão em transformar vendas em lucro, evidenciando o controle dos gastos e custos do dia a dia. Neves (2017) mostrou em seus estudos que analisar esses indicadores é vital, pois permite entender a fundo a situação da empresa e avaliar, de forma clara, como ela está se saindo. Essa compreensão ampla é a base para decidir os próximos passos. Utilizar métricas como o ROE, por exemplo, é importantíssimo para saber o retorno sobre o investimento, um ponto-chave para os acionistas. Em resumo, estudar o QICP, a Tx RCP e a MO dá uma visão completa da força, do lucro e da eficiência da Usina Coruripe Açúcar e Álcool.
	
	Além disso, O Grau de Endividamento Geral (GEG) é um indicador que mede a participação do capital de terceiros no financiamento dos ativos da empresa, ou seja, revela quanto do ativo total é sustentado por dívidas em vez de capital próprio. Esse índice é fundamental para avaliar o nível de dependência financeira da organização em relação a recursos externos, como empréstimos, financiamentos e obrigações com fornecedores. Um GEG elevado significa que uma parcela significativa dos ativos da empresa é financiada por terceiros; por outro lado, um GEG mais baixo indica que a empresa depende menos de dívidas e utiliza mais capital próprio, transmitindo maior solidez e segurança para investidores e credores. 
	
	Ademais, A Composição do Endividamento (CE) avalia a proporção das dívidas de curto prazo em relação ao total das obrigações da empresa. Esse indicador é relevante porque permite identificar se a maior parte do endividamento recai sobre compromissos de liquidez imediata ou se está distribuída ao longo do tempo, em dívidas de longo prazo. O EBIT (Earnings Before Interest and Taxes), é um indicador financeiro que mede a capacidade de geração de resultado de uma empresa apenas pelas suas atividades operacionais, sem considerar o impacto das despesas financeiras (juros de dívidas) e dos tributos sobre o lucro. Ele mostra, portanto, se o negócio é eficiente e lucrativo em sua operação principal, independentemente da forma de financiamento adotada ou da carga tributária incidente. 
	
	\section{INDICADORES, APLICABILIDADES E RESULTADOS}
	\subsection{Liquidez Corrente - LC}
	\hspace*{1.5cm} A forma para calcular a liquidez líquida, de acordo com Marques (2025), é possível pela seguinte fórmula:
	
		\begin{center}
			\begin{tabular}{|c|}
				\hline
				\left( \frac{ATIVO  \ CIRCULANTE}{PASSIVO \  CIRCULANTE} \right) = LC \\
				\hline
			\end{tabular}
		\end{center}
	Dessa forma, tendo como base o ano de 2023, obtemos:
	
		\begin{center}
			\begin{tabular}{|c|}
				\hline
				LC = (\left \frac{4.291.554}{5.266.579} \right) = 0,81 \\
				\hline
			\end{tabular}
		\end{center}
	Além disso, olhando para o ano de 2024, conseguimos obter o seguinte resultado:
	
		\begin{center}
			\begin{tabular}{|c|}
				\hline
				LC = (\left \frac{5.128.421}{6.823.968} \right) = 0,75 \\
				\hline
			\end{tabular}
		\end{center}
		
		Na mesma linha de pensamento de Marques (2025), afirma-se que um valor superior a 1 (ou 100\%) indica que a empresa Usina possui mais ativos de curto prazo do que obrigações de curto prazo, sinalizando maior solidez financeira. Dessa forma, como a liquidez corrente de 2024 é maior que a liquidez corrente de 2023, isso sugere que a empresa tem uma boa capacidade de honrar suas dívidas. Além disso, nota-se que houve uma variação positiva de 19\% nos ativos circulantes, possuindo mais recursos à disposição. Em relação ao passivo circulante, houve uma variação positiva de 29\%, significando um meno endividamento. Portanto, a usina está em uma posição confortável para lidar com suas obrigações imediatas. 
	
	\subsection{Margem Bruta - MB}
	
	\hspace*{1.5cm} A Margem Bruta (MB) é dada pela fórmula abaixo:
		\begin{center}
			\begin{tabular}{|c|}
				\hline
				MB = \left( 100 · \left[ \frac{LUCRO \, BRUTO}{RECEITA \, LIQUIDA} \right] \right)  \\
				\hline
			\end{tabular}
		\end{center}
	Portanto, com base no ano de 2023 obtemos o seguinte resultado:
		\begin{center}
			\begin{tabular}{|c|}
				\hline
				MB = \left( 100  \left[ \frac{1.749.674}{4.376.917} \right] \right) = 34,6\% \\
				\hline
			\end{tabular}
		\end{center}
	Além disso, com base no ano de 2024 chegamos ao seguinte resultado:
		\begin{center}
			\begin{tabular}{|c|}
				\hline
				MB = \left( 100  \left[ \frac{1.384.286}{4.928.662} \right] \right) = 28\% \\
				\hline
			\end{tabular}
		\end{center}
		
	A Margem Bruta é um indicador de lucratividade. Sendo assim, ela indica a lucratividade de uma empresa após a dedução dos custos diretos associados à produção ou venda de seus bens ou serviços. Portanto, um valor mais alto indica maior eficiência na produção e maior capacidade de cobrir outras despesas operacionais e gerar lucro. Assim, a diminuição de 34,6\% para 28\% indica que a usina se tornou menos eficiente em gerar lucros.
		
	\subsection{Retornos Sobre o Investimanto - ROA}
	
	\hspace*{1.5cm} O ROA (Return on Assets), ou Retorno sobre o Ativo, é um indicador financeiro que mostra a eficiência com que uma empresa usa seus ativos para gerar lucro. Ele é descrito pela seguinte fórmula:
		
		\begin{center}
			\begin{tabular}{|c|}
				\hline
				ROA = \left( 100 · \left[ \frac{LUCRO \, LIQUIDO}{ATIVO \, TOTAL} \right] \right)  \\
				\hline
			\end{tabular}
		\end{center}
		Sendo assim, para o ano de 2023, temos:
		\begin{center}
			\begin{tabular}{|c|}
				\hline
				ROA = \left( 100 · \left[ \frac{140.419}{10.400.403} \right] \right) = 1,35\% \\
				\hline
			\end{tabular}
		\end{center}
		Por outro lado, para o ano de 2024 podemos chegamos ao resultado: 
			\begin{center}
				\begin{tabular}{|c|}
					\hline
					ROA = \left( 100 · \left[ \frac{650.143}{10.878.562} \right] \right) = 5,97\% \\
					\hline
				\end{tabular}
			\end{center}
			
		Portanto, o cálculo do ROA para a Usina Coruripe, comparando 2023 e 2024, mostra uma melhora significativa no desempenho da empresa. Em 2023, o ROA foi de 1,35\%, indicando que a empresa gerou R\$ 0,0135 de lucro para cada real de ativo. Já em 2024, esse indicador subiu para 5,98\%. Isso significa que a empresa se tornou mais eficiente em usar seus ativos para gerar lucro, mostrando uma gestão mais eficaz no período de 2024.
		
	
	
	
	\subsection{Quociente de Imobilização do Capital Próprio - QICP}
	
	\hspace*{1.5cm} O Quociente de Imobilização do Capital Próprio é dado pela fórmula abaixo:
		\begin{center}
			\begin{tabular}{|c|}
				\hline
				QICP = \left( 100 · \left[ \frac{IMOBILIZADO}{PATRIMÔNIO LÍQUIDO} \right] \right)  \\
				\hline
			\end{tabular}
		\end{center}
		
	Sendo assim, para o ano de 2023, temos:
		
		\begin{center}
			\begin{tabular}{|c|}
				\hline
				QICP = \left( 100 · \left[ \frac{2.193.618}{2.859.092} \right] \right) = 76,7\% \\
				\hline
			\end{tabular}
		\end{center}

	Por outro lado, para o ano de 2024, foi possível obter o valor de:
			
		\begin{center}
			\begin{tabular}{|c|}
				\hline
				QICP = \left( 100 · \left[ \frac{2.369.749}{3.270.213} \right] \right) = 72,4\% \\
				\hline
			\end{tabular}
		\end{center}
		
	Portanto, a usina caiu de 0,767 em 2023 para 0,724 em 2024. A queda do QICP de 0,767 para 0,724 em 2024 é um sinal positivo de maior liquidez relativa do capital próprio e de possível melhoria na estrutura de investimentos da empresa. Segundo a análise de Flávio Leonel de Carvalho (2008) sobre indicadores de avaliação de desempenho, a análise do quociente de imobilização é essencial para entender como a empresa aloca seus recursos.
	 
	\subsection{Taxa de Rentabilidade do Capital Próprio - TxRCP}
	
	\hspace*{1.5cm}Sabendo que a TxRCP é da pela fórmula: 
	
		\begin{center}
			\begin{tabular}{|c|}
				\hline
				TxRCP = \left( 100 · \left[ \frac{LUCRO \, LIQUIDO}{ PAT. \, LÍQUIDA} \right] \right)  \\
				\hline
			\end{tabular}
		\end{center}
	Para o ano de 2023 é possível verificar o seguinte resultado:
	
	\begin{center}
		\begin{tabular}{|c|}
			\hline
			TxRCP = \left( 100 · \left[ \frac{140.419}{2.859.092} \right] \right) = 5\% \\
			\hline
		\end{tabular}
	\end{center}
	Além disso, para o ano de 2024 verificamos o seguinte resultado:
	
	\begin{center}
		\begin{tabular}{|c|}
			\hline
			TxRCP = \left( 100 · \left[ \frac{650.143}{3.270.213} \right] \right) = 19,88\% \\
			\hline
		\end{tabular}
	\end{center}

	A TxRCP, também conhecida como ROE, da empresa teve um aumento significativo, de 5\% em 2023 para 19,88\% em 2024. O aumento do ROE de 5\% para 19,88\% entre 2023 e 2024 é um sinal claro de melhora na rentabilidade da empresa.  Segundo a pesquisa de Juliana Medeiros das Neves (2017), a análise de indicadores como o ROE é crucial, pois demonstra a capacidade da empresa de gerar lucro em relação ao capital investido pelos sócios, sendo um dos fatores mais importantes para os acionistas.



	\subsection{Margem Operacional - MO}
	
	\hspace*{1.5cm} Esse índice mostra a parcela da receita líquida que se transforma em lucro operacional antes de juros e impostos, revelando o quão bem a empresa administra seus custos e despesas em relação ao faturamento. 
	 

	 
	 A Margem Operacional é dada pela fórmula:

		\begin{center}
			\begin{tabular}{|c|}
				\hline
				MO = \left( 100 · \left[ \frac{LUCRO \, OPERACIONAL}{ RECEITA \, LÍQUIDA} \right] \right)  \\
				\hline
			\end{tabular}
		\end{center}
		
	Então para o ano de 2023 obtemos o seguinte resultado: 
	
		\begin{center}
			\begin{tabular}{|c|}
				\hline
				MO = \left( 100 · \left[ \frac{713.652}{3.075.471} \right] \right) = 23\% \\
				\hline
			\end{tabular}
		\end{center}
		
	Além disso, para o ano de 2024 o seguinte resultado foi alcançado: 
	
		\begin{center}
			\begin{tabular}{|c|}
				\hline
				MO = \left( 100 · \left[ \frac{788.999}{3.493.657} \right] \right) = 22,5\% \\
				\hline
			\end{tabular}
		\end{center}
	
	A Margem Operacional da Usina Coruripe melhorou, caiu de 23\% em 2023 para 22,5\% em 2024. Mesmo com a queda de 0,5 p.p. na margem operacional, o resultado de 22,5\% ainda reflete uma operação sólida e eficiente. A análise de indicadores de desempenho, como apresentada por Flávio Leonel de Carvalho (2008), mostra que a queda na margem operacional sugere que a gestão de custos e despesas está menos eficaz.
	
	\subsection{Grau de Endividamento - CEG}
	
	\hspace*{1.5cm} Como o Grau de Endividamento é dado pela fórmula:
	
	\begin{center}
		\begin{tabular}{|c|}
			\hline
			\left( \frac{PASSIVO \, EXIGÍVEL}{TOTAL \, DE \, ATIVOS} \right) = CEG \\
			\hline
		\end{tabular}
	\end{center}
	Sendo assim, iremos calcular este indicador com base nos anos de 2023 e 2024. Portanto, para o ano de 2023, temos:
	
	\begin{center}
		\begin{tabular}{|c|}
			\hline
			CEG = \left( 100 · \left[ \frac{7.541.311}{10.400.403} \right] \right) = 72,50\% \\
			\hline
		\end{tabular}
	\end{center}
	Por outro lado, para o ano de 2024 obtemos o seguinte resultado:
	
	\begin{center}
		\begin{tabular}{|c|}
			\hline
			CEG = \left( 100 · \left[ \frac{7.893.627}{10.878.562} \right] \right) = 72,56\% \\
			\hline
		\end{tabular}
	\end{center}
	
	Dessa forma, em 2023, o Grau de Endividamento Geral da Usina Coruripe foi de 72,50\%, enquanto em 2024 o índice ficou praticamente estável, em 72,56\%. Isso significa que, em ambos os anos, cerca de quase três quartos dos ativos da empresa foram financiados por capital de terceiros, como empréstimos, financiamentos, arrendamentos e obrigações com fornecedores.
	
	A Usina Coruripe atua na produção de açúcar, etanol e energia, um setor que exige fortes investimentos em ativos fixos (máquinas, colheitadeiras, usinas, caldeiras, terrenos agrícolas, etc.), além de capital de giro elevado para sustentar o ciclo produtivo. Nesse contexto, empresas do setor apresentam índices de endividamento mais altos, já que precisam recorrer constantemente a financiamentos de longo prazo para renovar equipamentos, expandir a produção e suportar períodos de baixa no preço do açúcar e do etanol. O ponto positivo é que o GEG se manteve estável de 2023 para 2024, sem apresentar aumento expressivo, sugerindo que a empresa conseguiu manter o equilíbrio entre expansão e controle do endividamento.
	
	\subsection{Composição de Endividamento - CE}
	\hspace*{1.5cm} Sabendo que a Composição de Endividamento é dado por:
	
	\begin{center}
		\begin{tabular}{|c|}
			\hline
			\left(\frac{PASSIVO\, CIRCULANTE}{PASSIVO\,TOTAL} \right) = CE \\
			\hline
		\end{tabular}
	\end{center}
	Dessa forma, para o ano de 2023, temos:
	
	\begin{center}
		\begin{tabular}{|c|}
			\hline
			CE = \left( 100  \left[ \frac{3.418.785}{7.541.311} \right] \right) = 45,33\% \\
			\hline
		\end{tabular}
	\end{center}
	Por outro lado, para o ano de 2024 obtemos o seguinte resultado:
	
	\begin{center}
		\begin{tabular}{|c|}
			\hline
			CE = \left( 100 · \left[ \frac{2.787.400}{7.893.627} \right] \right) = 35,21\% \\
 			\hline
		\end{tabular}
	\end{center}
	
	Sendo assim, Em 2023, o índice foi de aproximadamente 45,33\%, revelando que quase metade das dívidas estava concentrada em compromissos de curto prazo, o que aumenta a pressão sobre a liquidez da empresa e pode exigir maior capacidade de geração de caixa imediato para evitar riscos financeiros. Já em 2024, o índice caiu para 35,31\%, mostrando um movimento de alongamento do perfil da dívida, com maior participação de obrigações de longo prazo. Essa mudança tende a reduzir o risco de descasamento entre os fluxos de caixa e os compromissos financeiros, proporcionando maior fôlego para o planejamento operacional e estratégico da companhia. Portanto, a estrutura temporal das dívidas da empresa, indica o peso relativo das obrigações de curto prazo sobre o total do endividamento.
	
	\subsection{Earnings Before Interest and Taxes - EBIT}
	
	\hspace*{1.5cm} Sabendo que Earnings Before Interest and Taxes em português conhecido como Lucro Antes dos Juros e Impostos (LAJIR) é dado por:
	
	\begin{center}
		\begin{tabular}{|c|}
			\hline
			EBIT = (RECEITA\,TOTAL - CUSTO\,DE\,PRODUTOS\,VENDIDOS - DESPESAS \,OPERACIONAIS )\\
			\hline
		\end{tabular}
	\end{center}
	Dessa forma, para o ano de 2023, temos:
		
	\begin{center}
		\begin{tabular}{|c|}
			\hline
			EBIT = [(3.078.997) - (2.166.875) - (198.470)] =  R\$ 713.652 \\
			\hline
		\end{tabular}
	\end{center}
	Além disso, para o ano de 2024 obtemos o seguinte resultado:
		
	\begin{center}
		\begin{tabular}{|c|}
			\hline
			EBIT = [(3.648.328) - (2.468.921) - (390.408)] =  R\$ 788.999 \\
			\hline
		\end{tabular}
	\end{center}
	
	Os resultados do EBIT mostram que a empresa manteve uma boa capacidade de geração de lucro operacional em ambos os períodos analisados. Em 2023, o EBIT foi de R\$ 713,6 bilhões, enquanto em 2024 houve um crescimento para R\$ 788,9 bilhões, indicando uma evolução operacional positiva mesmo diante de maiores despesas com vendas e administrativas. Esse aumento se deve, principalmente, ao ganho expressivo no resultado de participação societária em 2024, que compensou parcialmente os custos mais elevados. Dessa forma, observa-se que a empresa conseguiu ampliar sua eficiência operacional e reforçar sua lucratividade antes dos efeitos financeiros e tributários.

	\subsection{Margem Liquida}
	 
	\hspace*{1.5cm}De acordo com Silva (2007) e Marques (2020), esse indicador Relaciona o lucro líquido às vendas líquidas, indicando a lucratividade da organização. Pode ser influenciada por diversos fatores, como custos de produção e produtividadea. Ele representado pela fórmula:
	\begin{center}
		\begin{tabular}{|c|}
			\hline
			 MARGEM LÍQUIDA = \left( 100 · \left[ \frac{LUCRO \, LIQUIDO}{ VENDAS \, LÍQUIDAS} \right] \right)  \\
			\hline
		\end{tabular}
	\end{center}
	Dessa forma, para o ano de 2023, temos:	
	\begin{center}
		\begin{tabular}{|c|}
			\hline
			MARGEM LIQUIDA  = \left( 100 · \left[ \frac{140.419}{3.075.471} \right] \right) = 4.57\% \\
			\hline
		\end{tabular}
	\end{center}
	Por outro lado, é possível verificar que, no ano de 2024, obtemos o seguinte valor para esse indicador:
		\begin{center}
		\begin{tabular}{|c|}
			\hline
			MARGEM LIQUIDA  = \left( 100 · \left[ \frac{650.143}{3.493.657} \right] \right) = 18,61\% \\
			\hline
		\end{tabular}
	\end{center}
	
	Como em 2023, a margem líquida foi de 4,57\%, indicando que a empresa retinha uma pequena parte das receitas como lucro após custos e despesas. Por outro lado, em 2024, a margem subiu para 18,61\%, sugerindo uma melhora expressiva na gestão de custos e aumento na rentabilidade. Sendo assim, esse crescimento é um sinal positivo, indicando que a empresa se tornou mais eficiente em suas operações e tem potencial para um desempenho financeiro.
	
	
	
	\section{CONSIDERAÇÕES FINAIS}
	
	\newpage
	
	\begin{thebibliography}{9}
		
		\bibitem{assaf2006financas}
		Assaf Neto, A. (2006).
		\textit{Finanças Corporativas e Valor} (2nd ed.).
		São Paulo: Atlas.
		
	\end{thebibliography}
	
	% Definir cor clara e cor escura para destaques
	\definecolor{lightgray}{gray}{0.95}
	\definecolor{darkgray}{gray}{0.85}
	
	\newpage
	
	\centering \textbf{ANEXOS}
		
		\begin{center}
			\textbf{\Large S/A Usina Coruripe Açúcar e Álcool}\\
			\textbf{Balanço Patrimonial referente ao período de 31 de março de 2023 a 31 de dezembro de 2023 e de março de 2024 a 31 de dezembro de 2024}\\
			(Valores expressos em milhares de reais)
		\end{center}
		
		
		
		%==================== ATIVO ====================%
		\centering\section*{Ativo}
		Tabela 1 - Balanço patrimonial referente ao ano 2023 e 2024
		\rowcolors{2}{lightgray}{white} % Linhas zebras
		\begin{longtable}{p{6cm}r r r r }
			\toprule
			& \multicolumn{2}{c}{\textbf{Consolidado}} & \multicolumn{2}{c}{\textbf{Consolidado}} \\
			\cmidrule(lr){2-3} \cmidrule(lr){4-5}
			& 31/03/2023 & 31/12/2023 & 31/03/2024 & 31/12/2024 \\
			\midrule
			\endhead
			%---- Circulante
			\textbf{Circulante} & & & & \\
			Caixa e equivalentes de caixa & 246.682 & 390.862 & 769.658 & 1.155.469 \\
			Aplicações financeiras & 127.021 & 99.145 & 155.924 & 158.542 \\
			Contas a receber de clientes & 174.933 & 102.282 & 144.124 & 106.947 \\
			Estoques & 844.828 & 162.191 & 843.598 & 213.391 \\
			Adiantamentos a fornecedores & 248.367 & 217.172 & 250.313 & 210.817 \\
			Ativos biológicos & 525.165 & 486.996 & 0,00 & 562.940 \\
			Tributos a recuperar & 140.893 & 171.546 & 154.057 & 146.499 \\
			IR e CS a recuperar & 21.897 & 38.494 & 0,00 & 90.661 \\
			Partes relacionadas & 16.005 & 28.824 & 31.656 & 20.526 \\
			Instrumentos financeiros derivativos & 50.883 & 13.643 & 275.829 & 0,00 \\
			Outros direitos & 98.340 & 85.385 & 88.042 & 50.523 \\
			\rowcolor{darkgray}\textbf{Total do ativo circulante} & \textbf{2.495.014} & \textbf{1.796.540} & \textbf{2.395.349} & \textbf{2.733.072} \\
			\midrule
			%---- Não Circulante
			\textbf{Não circulante} & & & & \\
			Aplicações financeiras & 43.203 & 68.035& 8.478 & 1.525 \\
			Adiantamentos a fornecedores & 195.663 & 194.071 & 142.049 & 149.632 \\
			Partes relacionadas & 105 & 0,00 & 18.432 & 4.431 \\
			Tributos a recuperar & 4.098 & 5.052 & 5.142 & 4.431 \\
			IR e CS diferidos & 508.127 & 41.218 & 508.127 & 41.218 \\
			Instrumentos financeiros derivativos & 0,00 & 21.535 & 75.821 & 13.392 \\
			Depósitos judiciais & 4.524 & 6.251 & 6.974 & 6.391 \\
			Outros créditos/outros direitos & 4.035.181 & 4.216.551 & 4.467.192 & 4.273.838 \\
			Investimentos & 28.224 & 31.751 & 56.523 & 33.193 \\
			Imobilizado & 2.034.027 & 2.193.618 & 2.457.347 & 2.369.749 \\
			Intangível & 3.853 & 6.023 & 6.684 & 0,00 \\
			Direito de uso & 1.723.721 & 1.162.397 & 1.204.369 & 1.246.056 \\
			\rowcolor{darkgray}\textbf{Total do ativo não circulante} & \textbf{8.072.494} & \textbf{7.905.389} & \textbf{9.018.343} & \textbf{8.845.609} \\
			\midrule
			\rowcolor{darkgray}\textbf{Total do Ativo} & \textbf{9.869.034 } & \textbf{10.400.403} & \textbf{12.313.692} & \textbf{10.878.562} \\
			\bottomrule
		\end{longtable}
		
		
		
		
		\newpage
		%==================== PASSIVO E PL ====================%
		\centering\section*{Passivo e Patrimônio Líquido}
		Tabela 2 - Balanço patrimonial referente ao ano 2023 e 2024
		\rowcolors{2}{lightgray}{white} % Linhas zebras
		\begin{longtable}{p{6cm} r r r r}
			\toprule
			& \multicolumn{2}{c}{\textbf{Consolidado}} & \multicolumn{2}{c}{\textbf{Consolidado}} \\
			\cmidrule(lr){2-3} \cmidrule(lr){4-5}
			& 31/03/2023 & 31/12/2023 & 31/03/2024 & 31/12/2024 \\
			\midrule
			\endhead
			%---- Passivo Circulante
			\textbf{Passivo Circulante} & & & & \\
			Fornecedores & 200.066 & 539.767 & 621.204 & 335.828 \\
			Empréstimos e financiamentos & 904.387 & 1.595.291 & 1.743.158 & 1.295.309 \\
			Arrendamento a pagar & 146.348 & 88.011 & 166.196 & 145.323 \\
			Parceria agrícola a pagar & 182.891 & 190.933 & 196.693 & 226.012 \\
			Salários e encargos sociais & 76.272 & 97.077 & 105.753 & 81.723 \\
			Tributos a recolher & 25.137 & 34.636 & 32.455 & 34.256 \\
			IR e CS a pagar & 89,00 & 732,00 & 803.311 & 450.467 \\
			Adiantamento de clientes & 216.574 & 607.742 & 88.289 & 139.702 \\
			Compromissos com contrato de energia & 77.669 & 161.993 & 227.293 & 98.497 \\
			Instrumentos financeiros derivativos & 1.724 & 95.803 & 0 & 9.491 \\
			Outras obrigações & 16.637 & 6.800 & 22.236 & 9.491 \\
			\rowcolor{darkgray}\textbf{Total do passivo circulante} & \textbf{1.847.794} & \textbf{3.418.785} & \textbf{4.036.568} & \textbf{2.787.400} \\
			\midrule
			%---- Passivo Não Circulante
			\textbf{Passivo Não Circulante} & & & & \\
			Empréstimos e financiamentos & 2.737.544 & 2.257.269 & 2.784.617 & 3.050.316 \\
			Arrendamento a pagar & 843.717 & 347.909 & 418.251 & 456.455 \\
			Parceria agrícola a pagar & 556.067 & 508.406 & 599.871 & 548.577 \\
			Tributos a recolher & 168.868 & 180.305 & 176.765 & 12.485 \\
			Instrumentos financeiros derivativos & 44.327 & 0,00 & 13.392 & 75.821 \\
			Adiantamento de clientes & 254.296 & 108.503 & 532.633 & 268.787 \\
			Compromissos com contrato de energia & 140.355 & 33.233 & 25.419 & 19.945 \\
			IR e CS diferidos & 35.745 & 118.395 & 0,00 & 99.316 \\
			Provisões para contingências & 73.120 & 59.156 & 8.672 & 10.166 \\
			Outras obrigações & 487.711 & 509.350 & 516.787 & 564.359 \\
			\rowcolor{darkgray}\textbf{Total do passivo não circulante} & \textbf{5.341.750} & \textbf{4.122.526} & \textbf{5.175.723} & \textbf{5.106.227} \\
			\midrule
			\rowcolor{darkgray}\textbf{Total do Passivo} & \textbf{7.189.544} & \textbf{7.541.311} & \textbf{9.212.291} & \textbf{7.893.627} \\
			\midrule
			
			%---- Patrimônio Líquido
			\textbf{Patrimônio Líquido} & & & & \\
			Capital social & 408.845 & 408.845 & 867.567 & 867.567 \\
			Ações em tesouraria & (1.215) & (1.215) & (1.215) & (1.215) \\
			Ajuste de avaliação patrimonial & 26.987 & 104.773 & 37.464 & 248.494\\
			Reservas de lucros & 2.244.873 & 2.249.326 & 2.011.623 & 1.996.759 \\
			Lucros acumulados & 0,00 & 97.363 & 0.00 & 655.596 \\
			\rowcolor{darkgray}\textbf{Total do patrimônio líquido} & \textbf{2.679.490} & \textbf{2.859.092} & \textbf{2.915.439} & \textbf{3.270.213} \\
			\midrule
			\rowcolor{darkgray}\textbf{Total do Passivo e PL} & \textbf{9.869.034} & \textbf{10.400.403} & \textbf{10.878.562} & \textbf{12.313.692} \\
			\bottomrule
		\end{longtable}
		
\newpage
		
		\begin{center}
			\textbf{\Large S/A Usina Coruripe Açúcar e Álcool}\\
			\textbf{Demonstração do resultado períodos de três e nove meses findos, referente a 31 de dezembro de 2023 a 31 de dezembro de 2024.}\\
			(Valores expressos em milhares de reais, exceto quando indicado de outra forma)
		\end{center}
	
		\centering\section*{Demostração do Resultado}
		Tabela 3 - Demostração do Resultado referente ao ano 2023 e 2024
		\rowcolors{2}{lightgray}{white} % Linhas zebras
		\begin{longtable}{p{6cm}r r r r }
			\toprule
			& \multicolumn{2}{c}{\textbf{31 de dezembro de 2023}} & \multicolumn{2}{c}{\textbf{31 de dezembro de 2024}} \\
			\cmidrule(lr){2-3} \cmidrule(lr){4-5}
			& Trimestre & 9 meses & Trimestre & 9 meses \\
			\midrule
			\endhead
			Receitas operacional líquida & 1.301.446 & 3.075.471 & 1.435.005 & 3.493.657 \\
			Custos dos produtos vendidos & (853.218) &(2.006.538) & (1.071.095)& (2.473.281) \\
			\cmidrule(lr){2-3} \cmidrule(lr){4-5}
			\rowcolor{darkgray} \textbf{Lucro Bruto} & 448.228 & 1.068.993 & 363.910 & 1.020.376 \\
			\cmidrule(lr){2-3} \cmidrule(lr){4-5}
			Despesas com vendas & (50.981) & (160.337) & (67.124) &(209.488) \\
			Despesas gerais e administrativas & (58.822) & (177.119) & (56.407) & (180.920) \\
			Resultado de participação societária & 1.294 & 3.526 & 1.462 & 154.671 \\
			Outras receitas (despesas) operacionais, líquidas & (1.480) & (21.351) & 1.462 & 154. 671 \\
			\cmidrule(lr){2-3} \cmidrule(lr){4-5}
			\rowcolor{darkgray}\textbf{Lucro operacional} & 338.239 & 713.652 & 243.540 & 788.999 \\ 
			\cmidrule(lr){2-3} \cmidrule(lr){4-5}
			Receitas financeiras & 132.190 & 472.916 & 519.216 & 1.076.913 \\
			Despesas financeiras & (294.469) & (1.004.791) & (769.586) & (1.676.735)\\
			\cmidrule(lr){2-3} \cmidrule(lr){4-5}
			\rowcolor{darkgray}\textbf{Resultado financeiro} & (162.279) & (531.875) & (250.370) & (599.822) \\
			\cmidrule(lr){2-3} \cmidrule(lr){4-5}\\
			\rowcolor{darkgray}\textbf{Lucro (prejuízo) antes do imposto de renda e da contribuição social}  & 175.960 & 181.777 & (6.830) & 189.177 \\
			\cmidrule(lr){2-3} \cmidrule(lr){4-5}\\
			Imposto de renda e contribuição social correntes & (570) & (2.313) & (688) & (1.976)\\
			Imposto de renda e contribuição social diferidos & (27.673) & (39.045) & (22.181) & (462.942) \\
			\cmidrule(lr){2-3} \cmidrule(lr){4-5}
			\rowcolor{darkgray}\textbf{Resultado do período} & \textbf{147.717} & \textbf{140.419} & \textbf{29.699} & \textbf{650.143}\\
			\cmidrule(lr){2-3} \cmidrule(lr){4-5}\\
			Lucro (prejuízo) básico e diluído por ação & 105.51 & 100.30 & (21.21) & 464.39 \\
			\cmidrule(lr){2-3} \cmidrule(lr){4-5}
			
				
		\end{longtable}

\end{document}
